In our experiments we perform measurements on an ensemble of molecules that assume a classical distribution of initial quantum states. 
For simplicity we assume our possible initial states form a discrete set, so that we can describe our ensembles by specificying a pair 
$\left( \{ \ket{\psi_o} \}, P(\ket{\psi_o})\right)$ which defines the set of possible initial states $\{ \ket{\psi_o} \}$ and the fraction $P(\ket{\psi_o}$ of molecules in the ensemble initially occupying a given state $\ket{psi_o}$.
At a later time $t$ an initial state $\ket{\psi_o}$ will have evolved to the state $e^{H(t-t_o)/i\hbar}\ket{\psi_o} \equiv U(t,t_o) \ket{\psi_o} \equiv \ket{\psi,t}$, where $H$ is the hamiltonian, we which presume to be identical for each molecule in the hamiltonian.

If at time $t$ we measure some observable $A$ of each molecule, then we expect on average a result equal to the \emph{classical} expectation value $\dev{A}$ over the distribution $P(\ket{\psi_o})$ of the \emph{quantum} expectation value $\ev{A}(t) = \bra{\psi,t}A \ket{\psi,t}$, i.e.
$$
\dev{A} (t)= 
\sum_{\{\ket{\psi_o}\}} P(\ket{\psi_o}) \ev{A}(t)
$$
From an orthonormal basis $\{ \ket{n} \}$ we can construct and insert the identity $\sum_n \ket{n}\bra{n}$ to obtain the following alternative expression for $\dev{A}$ 
\begin{equation}
\begin{split}
\dev{A}(t)
& = 
\sum_{\{\ket{\psi_o}\}} P(\ket{\psi_o}) \bra{\psi,t} \sum_n \ket{n}\bra{n} A \ket{\psi,t}
\\ & = 
\sum_n \bra{n} A \left( \sum_{\{\ket{\psi_o}\}} P(\ket{\psi_o}) \ket{\psi,t} \bra{\psi,t} \right) \ket{n}
\\ & = 
\text{Tr} \left( A \rho(t) \right)
\end{split}
\label{eq:ev}
\end{equation}
Where 
\begin{equation}
\begin{split}
\rho(t) 
& \equiv 
\sum_{\{\ket{\psi_o}\}} P(\ket{\psi_o}) \ket{\psi,t} \bra{\psi,t} 
\\ & =
U(t,t_o) \left( \sum_{\{\ket{\psi_o}\}} P(\ket{\psi_o}) \ket{\psi_o} \bra{\psi_o} \right) U^\dagger (t,t_o)
\\ & \equiv
U(t,t_o) \rho_o U^\dagger (t,t_o)
\end{split}
\label{eq:do}
\end{equation}
is the \emph{density operator} at a time $t$ of an ensemble with a distribution $P(\ket{\psi}_o)$ at a time $t_o$.

To calculate any observable $A$ associated with an ensemble (defined at a time $t_o$) at some later time $t$ it is therefore sufficient to compute the ensemble's initial density operator $\rho(t_o) \equiv \rho_o$ and ``propagate'' the operator forward in time (i.e. $\rho_o \to \rho' = \rho(t)$) using equation \eqref{eq:do}.

Note that equation \eqref{eq:ev} also implies that two ensembles 
$\left( \{ \ket{\psi_o} \}, P(\ket{\psi_o}) \right)$ and $ \left( \{ \ket{\psi_o} \}', P'(\ket{\psi_o})\right)$ 
that have the same associated initial density operators $\rho_o \equiv \rho'_o$ are equivalent in that for all times they produce identical measurable predictions.

One can show that a density operator $\rho$ at any time
\begin{itemize}
    \item is hermitian
    \item has non-negative eigenvalues and
    \item has unit trace.
\end{itemize}
The converse is also true -- for any hermitean operator $\rho$ with non-negative eigenvalues and unit trace we can find an ensemble 
$\left( \{ \ket{\psi_o} \}, P(\ket{\psi_o}) \right)$ so that $\sum_{\{\ket{\psi_o}\}} P(\ket{\psi_o}) \ket{\psi_o} \bra{\psi_o} = \rho$. 
Therefore the study of ensembles of quantum mechanical states is reduced to the study of operators $\rho$ with the three above-mentioned properties evolving in time according to equation \eqref{eq:do}.